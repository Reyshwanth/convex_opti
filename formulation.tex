\documentclass{article}
\usepackage{amsmath}
\usepackage{amssymb}
\usepackage{geometry}
\geometry{a4paper, margin=1in}

\title{GCS Path Planning: Formulation and Decomposition}
\author{Antigravity}
\date{\today}

\begin{document}

\maketitle

\section{Introduction}
This document details the mathematical formulation for the Graph of Convex Sets (GCS) path planning algorithm and the convex decomposition method used to generate the free space regions.

\section{Convex Decomposition}
The goal is to decompose the free space $\mathcal{F} = \mathcal{W} \setminus \bigcup_{j=1}^M \mathcal{O}_j$, where $\mathcal{W}$ is the rectangular domain and $\mathcal{O}_j$ are convex obstacles, into a set of convex polygons $\{R_1, \dots, R_N\}$.

\subsection{Hole Integration}
Since the free space is multiply-connected (contains holes), we first transform it into a simply-connected polygon $P$.
\begin{enumerate}
    \item Let $P_0$ be the boundary of $\mathcal{W}$ (ordered Counter-Clockwise).
    \item For each obstacle $\mathcal{O}_j$ (ordered Clockwise):
    \begin{itemize}
        \item Find a "bridge" connecting a vertex $v \in \mathcal{O}_j$ to a visible vertex $u \in P_{current}$.
        \item Insert the sequence of $\mathcal{O}_j$ vertices into $P_{current}$ at $u$, effectively merging the hole into the boundary.
    \end{itemize}
    \item The result is a single polygon $P$ (possibly with self-touching edges) that represents $\mathcal{F}$.
\end{enumerate}

\subsection{Partitioning}
We decompose $P$ into convex regions using a heuristic approach based on resolving reflex vertices (vertices with internal angle $> 180^\circ$).

\textbf{Algorithm: Convex Decomposition}
\begin{enumerate}
    \item Initialize $Regions \gets \emptyset$.
    \item While $P$ is not convex:
    \begin{enumerate}
        \item Find a diagonal $d = (v_i, v_k)$ such that:
        \begin{itemize}
            \item The polygon formed by $v_i, \dots, v_k$ is convex.
            \item The diagonal $d$ does not intersect any other edges of $P$.
            \item No other vertices of $P$ lie inside the formed polygon.
        \end{itemize}
        \item Let $C = \{v_i, \dots, v_k\}$.
        \item $Regions \gets Regions \cup \{C\}$.
        \item $P \gets P \setminus C$ (Update boundary of $P$).
    \end{enumerate}
    \item $Regions \gets Regions \cup \{P\}$.
\end{enumerate}

\section{Trajectory Optimization (MICP)}

We solve a Mixed-Integer Convex Program (MICP) to find a path through the graph of convex sets and optimize the trajectory within them.

\subsection{Sets and Indices}
\begin{itemize}
    \item $V$: Set of convex regions (polygons), indexed by $i$.
    \item $E$: Set of edges $(i, j)$ representing adjacency between region $i$ and $j$.
    \item $K$: Degree of the B\'ezier curve (we use $K=2$ for quadratic).
    \item $d$: Dimension ($d=2$).
\end{itemize}

\subsection{Variables}
\begin{itemize}
    \item $y_i \in \{0, 1\}$: Binary variable, equal to 1 if region $i$ is visited.
    \item $z_{ij} \in \{0, 1\}$: Binary variable, equal to 1 if the transition from region $i$ to $j$ is active.
    \item $x_{i,k} \in \mathbb{R}^2$: Control point $k$ ($k \in \{0, \dots, K\}$) for the B\'ezier curve in region $i$.
    \item $t_{i,k} \in \mathbb{R}^2$: Slack variables for path length (velocity) minimization.
    \item $a_{i} \in \mathbb{R}^2$: Slack variables for acceleration (smoothness) minimization.
\end{itemize}

\subsection{Optimization Problem}

The objective is to minimize a weighted sum of path length and acceleration (smoothness).

\begin{equation*}
\min \sum_{i \in V} \left( \sum_{k=1}^K \| x_{i,k} - x_{i,k-1} \|_1 + \lambda_{smooth} \| x_{i,2} - 2x_{i,1} + x_{i,0} \|_1 \right)
\end{equation*}

\noindent \textbf{Subject to:}

\subsubsection*{1. Flow Conservation}
\begin{align*}
& \sum_{j: (s,j) \in E} z_{sj} - \sum_{j: (j,s) \in E} z_{js} = 1 \quad (\text{Start Node } s) \\
& \sum_{j: (g,j) \in E} z_{gj} - \sum_{j: (j,g) \in E} z_{jg} = -1 \quad (\text{Goal Node } g) \\
& \sum_{j: (i,j) \in E} z_{ij} - \sum_{j: (j,i) \in E} z_{ji} = 0 \quad \forall i \in V \setminus \{s, g\} \\
& y_i \ge \sum_{j} z_{ij}, \quad y_i \ge \sum_{j} z_{ji}
\end{align*}

\subsubsection*{2. Containment}
\begin{align*}
& A_i x_{i,k} \le b_i + M(1 - y_i) \quad \forall i \in V, k \in \{0,\dots,K\} \\
& -M y_i \le x_{i,k} \le M y_i \quad (\text{Force } x_{i,k}=0 \text{ if } y_i=0)
\end{align*}

\subsubsection*{3. Continuity ($C^0$)}
\begin{equation*}
\| x_{i,K} - x_{j,0} \|_\infty \le M(1 - z_{ij}) \quad \forall (i,j) \in E
\end{equation*}

\subsubsection*{4. Heading Consistency ($C^1$)}
To ensure smooth transitions, we enforce continuity of the velocity vector. For quadratic B\'ezier curves, the tangent is proportional to the difference between control points.
\begin{equation*}
\| (x_{i,K} - x_{i,K-1}) - (x_{j,1} - x_{j,0}) \|_\infty \le M(1 - z_{ij}) \quad \forall (i,j) \in E
\end{equation*}

\subsubsection*{5. Boundary Conditions}
\begin{equation*}
x_{s,0} = p_{\text{start}}, \quad x_{g,K} = p_{\text{goal}}
\end{equation*}

\vspace{1em}
\noindent \textbf{Note}: The L1 norm objective and infinity norm constraints are used to keep the problem linear (MILP), which is compatible with the HiGHS solver. $M$ is a sufficiently large constant ("Big-M"). $\lambda_{smooth}$ is a regularization weight to encourage smoother paths.

\end{document}
